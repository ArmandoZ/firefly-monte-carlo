%%%%%%%%%%%%%%%%%%%%%%%%%%%%%%%%%%%%%%%%%
% Jacobs Landscape Poster
% LaTeX Template
% Version 1.1 (14/06/14)
%
% Created by:
% Computational Physics and Biophysics Group, Jacobs University
% https://teamwork.jacobs-university.de:8443/confluence/display/CoPandBiG/LaTeX+Poster
%
% Further modified by:
% Nathaniel Johnston (nathaniel@njohnston.ca)
%
% This template has been downloaded from:
% http://www.LaTeXTemplates.com
%
% License:
% CC BY-NC-SA 3.0 (http://creativecommons.org/licenses/by-nc-sa/3.0/)
%
%%%%%%%%%%%%%%%%%%%%%%%%%%%%%%%%%%%%%%%%%

%----------------------------------------------------------------------------------------
%	PACKAGES AND OTHER DOCUMENT CONFIGURATIONS
%----------------------------------------------------------------------------------------

\documentclass[final]{beamer}

\usepackage[scale=1.24]{beamerposter} % Use the beamerposter package for laying out the poster
\usepackage{hyperref}

\usetheme{confposter} % Use the confposter theme supplied with this template

\setbeamercolor{block title}{fg=ngreen,bg=white} % Colors of the block titles
\setbeamercolor{block body}{fg=black,bg=white} % Colors of the body of blocks
\setbeamercolor{block alerted title}{fg=white,bg=dblue!70} % Colors of the highlighted block titles
\setbeamercolor{block alerted body}{fg=black,bg=dblue!10} % Colors of the body of highlighted blocks
% Many more colors are available for use in beamerthemeconfposter.sty

%-----------------------------------------------------------
% Define the column widths and overall poster size
% To set effective sepwid, onecolwid and twocolwid values, first choose how many columns you want and how much separation you want between columns
% In this template, the separation width chosen is 0.024 of the paper width and a 4-column layout
% onecolwid should therefore be (1-(# of columns+1)*sepwid)/# of columns e.g. (1-(4+1)*0.024)/4 = 0.22
% Set twocolwid to be (2*onecolwid)+sepwid = 0.464
% Set threecolwid to be (3*onecolwid)+2*sepwid = 0.708

\newlength{\sepwid}
\newlength{\onecolwid}
\newlength{\twocolwid}
\newlength{\threecolwid}
\setlength{\paperwidth}{48in} % A0 width: 46.8in
\setlength{\paperheight}{36in} % A0 height: 33.1in
\setlength{\sepwid}{0.024\paperwidth} % Separation width (white space) between columns
\setlength{\onecolwid}{0.22\paperwidth} % Width of one column
\setlength{\twocolwid}{0.464\paperwidth} % Width of two columns
\setlength{\threecolwid}{0.708\paperwidth} % Width of three columns
\setlength{\topmargin}{-0.5in} % Reduce the top margin size
%-----------------------------------------------------------

\usepackage{graphicx}  % Required for including images

\usepackage{booktabs} % Top and bottom rules for tables

%----------------------------------------------------------------------------------------
%	TITLE SECTION
%----------------------------------------------------------------------------------------

\title{Firefly Monte Carlo: Exact MC with data subsets} % Poster title

\author{Feynman Liang, Max Chamberlin, Kai Xu} % Author(s)

\institute{Cambridge University Engineering Department} % Institution(s)

%----------------------------------------------------------------------------------------

\begin{document}

\addtobeamertemplate{block end}{}{\vspace*{2ex}} % White space under blocks
\addtobeamertemplate{block alerted end}{}{\vspace*{2ex}} % White space under highlighted (alert) blocks

\setlength{\belowcaptionskip}{2ex} % White space under figures
\setlength\belowdisplayshortskip{2ex} % White space under equations

\begin{frame}[t] % The whole poster is enclosed in one beamer frame

\begin{columns}[t] % The whole poster consists of three major columns, the second of which is split into two columns twice - the [t] option aligns each column's content to the top

\begin{column}{\sepwid}\end{column} % Empty spacer column

\begin{column}{\onecolwid} % The first column

%----------------------------------------------------------------------------------------
%	OBJECTIVES
%----------------------------------------------------------------------------------------

\begin{alertblock}{Objectives}

Lorem ipsum dolor sit amet, consectetur, nunc tellus pulvinar tortor, commodo eleifend risus arcu sed odio:
\begin{itemize}
\item Mollis dignissim, magna augue tincidunt dolor, interdum vestibulum urna
\item Sed aliquet luctus lectus, eget aliquet leo ullamcorper consequat. Vivamus eros sem, iaculis ut euismod non, sollicitudin vel orci.
\item Nascetur ridiculus mus.
\item Euismod non erat. Nam ultricies pellentesque nunc, ultrices volutpat nisl ultrices a.
\end{itemize}

\end{alertblock}

%----------------------------------------------------------------------------------------
%	INTRODUCTION
%----------------------------------------------------------------------------------------

\begin{block}{Background}

A central quantity of interest when performing inference is the likelihood of
iid data $X = \{x_i\}_{i=1}^N$ under some parametric probability model with
parameters $\theta$:
\begin{equation}
  \label{eq:lik}
  L(\theta) = \prod_{i=1}^N L_n(\theta) = \prod_{i=1}^N P(x_i | \theta)
\end{equation}

For example, using Metroplis-Hastings MCMC to sample a posterior
distribution $P(\theta | X)$ requires computing the acceptance probability
\begin{align}
  A(\theta \to \theta')
  &= \min \left\{
      1,
      \frac{q(\theta' \to \theta) P(\theta'|X)}{q(\theta \to \theta') P(\theta | X)}
    \right\}\\
  &= \min \left\{
      1,
      \frac{q(\theta' \to \theta) L(\theta')P(\theta')}{q(\theta \to \theta') L(\theta)P(\theta)}
    \right\}
\end{align}
where the proposal distribution $q(\theta' \to \theta)$ is the transition
kernel for a reversible Markov chain.

Unfortunately, in modern web-scale datasets $N$ may be on the order of
petabytes and computing $L(\theta)$ at every iteration of the MC may
be prohibitively slow.

\end{block}

%----------------------------------------------------------------------------------------

\end{column} % End of the first column

\begin{column}{\sepwid}\end{column} % Empty spacer column

\begin{column}{\twocolwid} % Begin a column which is two columns wide (column 2)

\begin{columns}[t,totalwidth=\twocolwid] % Split up the two columns wide column

\begin{column}{\onecolwid}\vspace{-.6in} % The first column within column 2 (column 2.1)

%----------------------------------------------------------------------------------------
%	MATERIALS
%----------------------------------------------------------------------------------------

\begin{block}{Theory}

% TODO: discuss how FlyMC is ``exact'', augmenting joint does not change marginal
An auxillary variable method works by introducing auxillary variables $Z$
to form the joint distribution $P(X,Z|\theta) = P(Z|X,\theta) L(\theta)$,
then later marginalizing to yield $L(\theta) = \sum_Z P(X,Z|\theta)$.
In FlyMC, the auxillary \emph{brightness} variables $z_i$ are Bernoulli
\begin{equation}
  \label{eq:brightness}
  P(z_n | x_n, \theta)
  = \left(
  \frac{L_n(\theta) - B_n(\theta)}{L_n(\theta)}
  \right)^{z_n} \left(
  \frac{B_n(\theta)}{L_n(\theta)}
  \right)^{1-z_n}
\end{equation}
where $0 < B_n(\theta) \leq L_n(\theta)$ is a lower bound on the likelihood.
This yields
\begin{align}
  \label{eq:complete-lik}
  P(x_n,z_n|\theta) = \begin{cases}
    L_n(\theta) - B_n(\theta) & z_n = 1\\
    B_n(\theta) & \text{otherwise}
  \end{cases}
\end{align}
% TODO: interpret z_i as blinking fireflies
The key observation is that Equation~\ref{eq:complete-lik} \textbf{only
requires computation of $L_n(\theta)$ for the \emph{bright} points $\{x_n \in
X : z_n = 1\}$}. Similarly defining the \emph{dim} points $X \setminus
bright$ allows us to decompose
\begin{align}
  \label{eq:complete-lik-all-data}
  P(X,Z|\theta)
  &= \prod_{n=1}^N (L_n(\theta) - B_n(\theta))^{z_n} (B_n(\theta))^{1 - z_n} \\
  &= \prod_{n \in bright} (L_n(\theta) - B_n(\theta)) \prod_{n \in dim} B_n(\theta)
\end{align}

If $B_n(\theta)$ is chosen appropriately (e.g.\ exponential family) then
$\prod_{n \in dim} B_n(\theta)$ can be computed in constant time, resulting in
an overall $O(\frac{E[\# bright]}{N})$ speedup.

%TODO: discuss how to choose B_N, tightness at MAP estimate

\end{block}

\begin{block}{Algorithm}

\begin{enumerate}
  \item Initalize $\theta^{(0)}$.
  \item Sample $z^{(t)}_n \sim \text{Bernoulli}\left(
    \frac{L_n(\theta^{(t)}) - B_n(\theta^{(t)})}{L_n(\theta^{(t)})}
    \right)$
  \item Propose $\theta^{(t+1)} \sim q(\theta^{(t)} \to \theta^{(t+1)})$.
  \item Compute acceptance probability $A(\theta^{(t)} \to \theta^{(t+1})$
    with $z^{(t)}_n$ fixed and accept with probability $A(\theta^{(t)} \to \theta^{(t+1})$
\end{enumerate}

Many $z_n = 0$ so it is wasteful to directly sample the brightness variables
$z$. another MH MCMC to efficiently sample $z_n$ from
Equation~\ref{eq:brightness}. This yields \emph{implicit sampling} and
introduces an additional parameter $q_{dim \to bright}$ used as the proposal
probabilities for the brightness MC.

\end{block}

\end{column} % End of column 2.1

\begin{column}{\onecolwid}\vspace{-.6in} % The second column within column 2 (column 2.2)

\begin{block}{Logistic Regression}

\begin{figure}
\includegraphics[width=0.8\linewidth]{placeholder.jpg}
\caption{Figure caption}
\end{figure}

\begin{figure}
\includegraphics[width=0.8\linewidth]{placeholder.jpg}
\caption{Figure caption}
\end{figure}

\begin{table}
\vspace{2ex}
\begin{tabular}{l l l}
\toprule
\textbf{Treatments} & \textbf{Response 1} & \textbf{Response 2}\\
\midrule
Treatment 1 & 0.0003262 & 0.562 \\
Treatment 2 & 0.0015681 & 0.910 \\
Treatment 3 & 0.0009271 & 0.296 \\
\bottomrule
\end{tabular}
\caption{Table caption}
\end{table}

\begin{figure}
\includegraphics[width=0.8\linewidth]{placeholder.jpg}
\caption{Figure caption}
\end{figure}

\end{block}

\end{column} % End of column 2.2

\end{columns} % End of the split of column 2 - any content after this will now take up 2 columns width

\end{column} % End of the second column

\begin{column}{\sepwid}\end{column} % Empty spacer column

\begin{column}{\onecolwid} % The third column

\begin{block}{Softmax Regression}

\begin{figure}
\includegraphics[width=0.8\linewidth]{placeholder.jpg}
\caption{Figure caption}
\end{figure}

\begin{figure}
\includegraphics[width=0.8\linewidth]{placeholder.jpg}
\caption{Figure caption}
\end{figure}

\end{block}


%----------------------------------------------------------------------------------------
%	CONCLUSION
%----------------------------------------------------------------------------------------

\begin{block}{Conclusion}

\begin{enumerate}
  \item Investigation of how to choose $B_n$ such that $\prod_n B_n$ easy
  \item MAP estimation to optimize $B_n$ doesn't work when $N$ is large
  \item Implement on software platforms supporting large-scale data
\end{enumerate}

\end{block}


\end{column} % End of the third column

\end{columns} % End of all the columns in the poster

\end{frame} % End of the enclosing frame

\end{document}
